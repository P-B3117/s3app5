Lorsqu'on augmente la fréquence du générateur d'ondes jusqu'à 10GHz, la réflection du signal est beacoup plus forte, et les signaux d'entrée et de sortie sont non reconnaissables.
Les cartes réseaux ne peuvent pas décoder ces signaux, étant donnée que la longeur des impulsions est trop courte.
En effet, si les impuslions ne sont pas assez longue, comme dans le cas d'un signal de 1GHz, il est impossible pour la carte réseau de les identifier.
De plus, il est important que l'amplitude de l'onde soit détéctable et qu'il y ai une assez grande distinction entre 0 et 1, ce qui est difficile lorsque la réflexion du fil arrondit le signal.
