 Au niveau fréquentiel, les problèmes survenants sont surtouts liés au principe de l'impédance ramenée. Cela fait, que la réflection crée des annulations
 partielles trop importantes. Cela vient diminuer l'amplitude du signal jusqu'à un point ou la carte ne peut plus lire le signal émis (amplitude
 plus basse que 0.5V). Ce phénomène se produit lorsque la longueur d'un cable est proche d'un multiple de la longeur d'onde du signal.
 Ces problèmes peuvent être rêglés en adaptant l'impédance du circuit (rajouter des fin de connections sur les fils non-utilisés), en s'assurant
 que les fils possèdent une bonne longueur selon la fréquence souhaitée, ou encore en utilisant des cables spécialisés possédants déjà l'impédance
 souhaitée.