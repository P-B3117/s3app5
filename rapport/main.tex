\documentclass[DIV=15,paper=letter,titlepage=true,fontsize=12pt,headings=normal,captions=nooneline]{scrartcl}
\usepackage{Style}
\usepackage[justification=centering]{caption}

\begin{document}
% !TEX root = Main.tex% !TEX root = Main.tex% !TEX root = Main.tex% !TEX root = Main.tex
\begin{center}
{\bfseries\sffamily \textls[150]{Université de Sherbrooke}\\[4pt] \textls[75]{Faculté de génie}\\\textls[75]{Département de génie électrique et de génie informatique}}
\setcounter{page}{0}
\pagenumbering{roman}
\thispagestyle{empty}
\vfill
\textbf{\huge{Rapport d'app7: Space-Invaders}}

\vfill
\textsf{\large Interfaces utilisateurs graphiques}

\vfill
\textsf{Présenté à}\\
\textsf{L'équipe professorale}

\vfill
\vfill
\vfill
\textsf{\footnotesize remis le 12 avril 2024}

\end{center}


\noindent
\rule{\linewidth}{.8pt}

\noindent
Poulin-Bergevin, Charles\hfill Pouc1302\\
Stephenne, Laurent \hfill stel2002\\

\newpage
\renewcommand*\contentsname{Table des Matières}
\tableofcontents
\newpage

\setcounter{page}{0}
\pagenumbering{arabic}

 \section{Mise en Contexte} %\label{Introduction}
 \subsection{Introduction}
 La communication par réseau filaire, et plus spécifiquement coaxiaux, est encore
extrêmenent répandue et essentielle dans plusieurs milieus.
 \FloatBarrier

\section{Reproduction du Problème}
\subsection{Indiquer le schéma électrique}
\input{}
\FloatBarrier
\subsection{Comparer le signal reçu aux spécifications de la carte}
\input{}
\FloatBarrier

\section{Analyse temporelle}
\subsection{Identification de chaque créneau}
\input{}
\FloatBarrier
\subsection{Explication du Problème}
\input{}
\FloatBarrier
\subsection{Mesure par analyse temporelle des branches}
\input{}
\FloatBarrier


\section{Analyse fréquentielle}
\subsection{Explication du problème dans le domaine fréquentiel}
\input{}
\FloatBarrier
\subsection{Détermination précise des longueurs des 3 branches}
\input{}
\FloatBarrier


\section{Solution du problème observé}
\subsection{Solution simple sans modifier le réseau}
\input{}
\FloatBarrier
\subsection{Solution en remplaçant le connecteur en T}
\input{}
\FloatBarrier


\section{Viabilité de la technologie}
\subsection{Problèmes à 1GHz}
\input{}
\FloatBarrier
\subsection{Est-ce qu'un réseau avec des centaines de clients fontcionne en full duplex?}
\input{}
\FloatBarrier


\end{document}