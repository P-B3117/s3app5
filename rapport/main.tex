\documentclass[DIV=15,paper=letter,titlepage=true,fontsize=12pt,headings=normal,captions=nooneline]{scrartcl}
\usepackage{Style}
\usepackage[justification=centering]{caption}

\begin{document}
% !TEX root = Main.tex% !TEX root = Main.tex% !TEX root = Main.tex% !TEX root = Main.tex
\begin{center}
{\bfseries\sffamily \textls[150]{Université de Sherbrooke}\\[4pt] \textls[75]{Faculté de génie}\\\textls[75]{Département de génie électrique et de génie informatique}}
\setcounter{page}{0}
\pagenumbering{roman}
\thispagestyle{empty}
\vfill
\textbf{\huge{Rapport d'app7: Space-Invaders}}

\vfill
\textsf{\large Interfaces utilisateurs graphiques}

\vfill
\textsf{Présenté à}\\
\textsf{L'équipe professorale}

\vfill
\vfill
\vfill
\textsf{\footnotesize remis le 12 avril 2024}

\end{center}


\noindent
\rule{\linewidth}{.8pt}

\noindent
Poulin-Bergevin, Charles\hfill Pouc1302\\
Stephenne, Laurent \hfill stel2002\\

\newpage
\renewcommand*\contentsname{Table des Matières}
\tableofcontents
\newpage

\setcounter{page}{0}
\pagenumbering{arabic}

 \section{Mise en Contexte} %\label{Introduction}
 \subsection{Introduction}
 La communication par réseau filaire, et plus spécifiquement coaxiaux, est encore
extrêmenent répandue et essentielle dans plusieurs milieus.
%  \FloatBarrier
 \subsection{Mise en situation}
 allo
%  \FloatBarrier
% \section{Présentation du jeu} \label{Présentation du jeu}
% \input{Partie 2/Application}
% \FloatBarrier
% \subsection{Menu} \label{Menu}
% \input{Partie 2/Menu}
% \FloatBarrier
% \subsection{Paramètres} \label{Parametre}
% \input{Partie 2/Parametre}
% \FloatBarrier
% \subsection{Jeu} \label{Jeu}
% \input{Partie 2/Jeu}
% \FloatBarrier
% \section{Ergonomie} \label{Ergonomie}
% \FloatBarrier
% \subsection{Mise en contexte} \label{Mise_en_contexte}
% \input{Partie 3/Mise en contexte}
% \FloatBarrier
% \subsection{Menu} \label{Ergo_Menu}
% \input{Partie 3/Ergo_Menu}
% \FloatBarrier
% \subsection{Paramètre} \label{Ergo_Parametre}
% \input{Partie 3/Ergo_Parametre}
% \FloatBarrier
% \subsection{Jeu} \label{Ergo_Jeu}
% \input{Partie 3/Ergo_Jeu}
% \section{Tests et vérifications de l'application} \label{tests et vérif}
% \subsection{Mise en contexte} \label{test_contexte}
% \input{Partie 4/Mise en contexte}
% \FloatBarrier
% \subsection{Plan de test} \label{test_plan}
% \input{Partie 4/Plan de test}
% \FloatBarrier
% \section{diagrammes} \label{diagrammes}
% \subsection{Mise en contexte} \label{diag_contexte}
% \input{Partie 5/Mise en contexte}
% \FloatBarrier
% \subsection{Diagramme de classe} \label{diag_classe}
% \input{Partie 5/Diagramme de classe}
% \FloatBarrier
% \subsection{Diagramme de cas d'utilisation} \label{diag_util}
% \input{Partie 5/Diagramme de cas d'utilisation}
% \FloatBarrier




\end{document}